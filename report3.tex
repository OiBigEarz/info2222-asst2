\documentclass[12pt,a4paper]{article}
\usepackage{graphicx} % for including images
\usepackage[utf8]{inputenc}
\usepackage{geometry}
\geometry{a4paper, margin=1in}
\usepackage{hyperref} % for hyperlinks

\title{Info2222 Assignment 2 report}
\date{}

\begin{document}
\maketitle

\section{Introduction}
% Brief introduction to your project and what this document will cover

\subsection{User Persona}
\subsubsection*{Persona Overview}

% Including the persona information as a table
\noindent
\begin{tabular}{|p{0.3\linewidth}|p{0.6\linewidth}|}
\hline
\textbf{Attribute} & \textbf{Details} \\
\hline
\textbf{Photo Name:} & Sam Hong \\
\hline
\textbf{Gender:} & Male \\
\hline
\textbf{Current Role:} & Student \\
\hline
\textbf{Age:} & 20 years \\
\hline
\textbf{Education/Background:} & Second year University of Sydney Computer Science \\
\hline
\textbf{Area of Interest:} & Cybersecurity, AI, and data science \\
\hline
\textbf{Goal and Tasks:} & To efficiently collaborate on group for academic purposes, and being able to message friends and family \\
\hline
\textbf{Environment:} & Owns a desktop and smart phone, has daily uses for messaging applications \\
\hline
\textbf{Quote:} & \textit{"For academic messaging application, I think, as simple desing like a whatsapp is best"} \\
\hline
\end{tabular}

\section{Introduction}
This section presents an analysis of the data collected through a Google Form survey distributed among peers at the University of Sydney. The purpose of the survey was to gather insights to create a detailed persona, which is seen from the table above for our project on enhancing the messaging system used by students.

\section{Methodology}
The survey was shared using an online Google Form, accessible at \url{https://forms.gle/76AmQiAeyXHEHCMZ7}. It aimed to collect diverse opinions and experiences related to the current messaging systems in use by the students.

\section{Demographic and Usage Analysis}
The majority of respondents were male students in their second and third years of study, which provides a clear demographic focus for our user persona. In terms of technology, most respondents reported owning smartphones and desktops, indicating high accessibility to digital platforms. The preferred messaging apps among the participants were Instagram and Discord, which are primarily used for communication between family, friends, and for academic purposes. Furthermore, daily usage of these apps was reported, suggesting a high dependency on digital communication. Regarding navigation preferences, a significant number of users preferred using both mouse and keyboard, which should be considered in designing the interface for enhanced usability.

\section{Feedback Analysis}
\subsection{Features Lacking in Current Systems}
Participants highlighted several features missing from current messaging systems that could enhance their academic experience:
\begin{itemize}
    \item Customization options such as themes, fonts, and layouts to reduce eye strain.
    \item Better organization features like dedicated channels for different subjects or projects.
    \item Enhanced search functions.
    \item Overall sleeker design.
\end{itemize}

\subsection{Challenges Faced}
Respondents identified key challenges with current messaging systems:
\begin{itemize}
    \item Cluttered user interfaces making navigation difficult.
    \item Issues with responsiveness on different devices.
    \item Difficulty tracking important messages in large group chats.
    \item Slow loading times and lack of integration with other academic tools.
\end{itemize}

\subsection{Suggested Improvements}
Based on the limitations identified, the following improvements were suggested:
\begin{itemize}
    \item A cleaner, more minimalist design with dropdown menus for complex functions.
    \item Options to tag messages or create threads within chats.
    \item Integration of notifications or reminders for academic deadlines.
    \item Faster loading times and simplified user interface.
\end{itemize}

\end{document}
